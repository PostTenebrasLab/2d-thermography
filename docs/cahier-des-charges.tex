\documentclass[10pt,a4paper]{article}
\usepackage[utf8]{inputenc}
\usepackage[francais]{babel}
\usepackage[T1]{fontenc}
\usepackage{fancyhdr} % Titre courant
\usepackage{amsmath}
\usepackage{amsfonts}
\usepackage{amssymb}
\usepackage{graphicx}
\usepackage{color}
\usepackage{multicol}
\usepackage{listings}
\usepackage{authblk}
\usepackage[left=2cm,right=2cm,top=2cm,bottom=2cm]{geometry}
\definecolor{dkgreen}{rgb}{0,0.4,0}
\definecolor{dkred}{rgb}{0.4,0,0}
\definecolor{gray}{rgb}{0.5,0.5,0.5}
\definecolor{lightgray}{rgb}{0.9,0.9,0.9}
\usepackage{lastpage} % pour avoir \pageref{LastPage} : le nombre total de pages du doc
\usepackage{wasysym}   % smiley, frownie

%% Packages pour les tableaux
\usepackage{array}     % Outils 0suppl<E9>mentaires
\usepackage{multirow}  % Colonnes multiples
\usepackage[table]{xcolor}
\usepackage{tabularx}  % Largeur totale donne
\usepackage{longtable} % sur plusieurs pages

%supprime toute indentation de paragraphe
\setlength{\parindent}{0cm}


\title{Thermographie 2d }
\author{Sebastien Chassot \\ sebastien.chassot@etu.hesge.ch}
\affil{HES-SO - hepia - section ITI \\ Université d'été}

\usepackage{hyperref}
\hypersetup{colorlinks=true, linkcolor=blue}
\hypersetup{pdfauthor=Sebastien Chassot}
\hypersetup{pdfstartpage=1}
\hypersetup{pdfpagemode=None} %FullScreen, None
\hypersetup{pdfpagelayout=SinglePage} %SinglePage, OneColumn, TwoColumnLeft, TwoColumnRight
\hypersetup{pdfstartview=FitH} %Fit, FitH, FitV, FitB, FitBH, FitBV
\pdfcompresslevel=9

%infos sur le document : auteur, titre, version (utilis<E9>es en page de garde, pied de page, filigrane...)
\newcommand{\docauthor}{Sebastien Chassot}
\newcommand{\docversion}{}
\newcommand{\doctitle}{Université d'été - Thermographie 2d }
%date du document
%\newcommand{\docdateddmmyy}{31/05/2013}
%\newcommand{\docdatemmyy}{05/2013}
\newcommand{\class}{hepia ITI}
%----------------------------------------------------------

% Entte et pied de page.
\pagestyle{fancy}
\renewcommand{\footrulewidth}{1pt}
%\renewcommand{\headrulewidth}{1pt}
\fancyfoot[C]{\scriptsize\emph{\class}}
\fancyfoot[R]{\scriptsize\emph{Page~\thepage~sur~\pageref{LastPage}}}
\fancyfoot[L]{\scriptsize\emph{\docauthor{}}}
%\footheight{100pt}
%\setlength{\headheight}{50pt} 
%%\renewcommand*\contentsname{\vspace{0.5cm}}
%\fancyhead[L]{Sebastien Chassot}
\fancyhead[C]{\textsc{Thermographie 2d}}
%\fancyhead[R]{hepia ITI}

%\fancypagestyle{plain}{%
%\fancyhf{} % clear all header and footer fields
%\fancyfoot[C]{} % except the center
%\renewcommand{\headrulewidth}{1pt}
%\renewcommand{\footrulewidth}{0pt}
%\fancyhead[L]{Sebastien Chassot}
%\fancyhead[C]{\textsc{TP introduction Matlab}}
%\fancyhead[R]{hepia ITI}
%}

\begin{document}
\maketitle



%%\tableofcontents

\section*{Introduction}

Le but de ce travail est de générer un thermogramme à l'aide d'un thermomètre infrarouge.\\

Deux moteurs pas-à-pas permettent d'orienter un capteur IR en 2 dimensions. Ce mécanisme permet de scanner une surface (X-Y). Pour chaque coordonnée, une mesure de température est effectuée. Au final, les données permettent de générer une cartographie thermique de la surface sous forme d'image (ou thermogramme).

\section*{Cahier des charges}

Le projet se subdivise en 3 parties :

\begin{enumerate}
\item La partie matérielle avec un micro-contrôleur gérant le positionnement du capteur, les mesures et la transmission vers un PC.
\item Le flux de données reçues est stocké dans un fichier.
\item Un traitement mathématique pour générer une image.
\end{enumerate}

\subsection*{Matériel}

Un arduino gère le positionnement du capteur, la mesure et la transmission vers un PC.

Le but est d'avoir un positionnement le plus précis possible (gestion du micro stepping), d'éviter les vibrations lors des déplacements et de gérer les accélérations tout en synchronisant la mesure.

Le thermomètre a deux modes de fonctionnement. Il peut transmettre les données via un bus 2wire ou en continu en PWM (10bits). Trouver le meilleur moyen d'acquérir une mesure fiable.

Transmission via un lien série.


\subsection*{Logiciel}

Le but est choisir les données à acquérir depuis un PC et de récupérer le flux de donnée, éventuellement le remettre "dans l'ordre" et les stocker dans un fichier.

\subsection*{Traitement d'image}

Avec octave afficher les données év, utiliser un algorithme pour améliorer la résolution et le contraste.

%%% Body



\end{document}
